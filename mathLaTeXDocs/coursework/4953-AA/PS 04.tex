\documentclass[letter, 12pt]{article}
\usepackage{comment} % enables the use of multi-line comments (\ifx \fi) 
\usepackage{fullpage} % changes the margin
\usepackage[T1]{fontenc}
\usepackage{selinput}
\usepackage{enumitem}
\usepackage{listings}
\usepackage{amsmath}
\usepackage{amssymb}
\usepackage{setspace}

\newcommand{\Mod}[1]{\ (\mathrm{mod}\ #1)}

\begin{document}
%Header-Make sure you update this information!!!!
\noindent
\large\textbf{Modern Abstract Algebra II} \hfill \textbf{Quinn Murphey} \\
\normalsize MAT 4953.001 \hfill Date:2/16/19 \\
Dr. Patterson \hfill Due Date: 02/15/19 \\
\noindent\makebox[\linewidth]{\rule{\paperwidth}{0.4pt}}

\section*{Problem Set 4}
\doublespacing
\textbf{Problem 1:}
    
    Finish the proof that if $F$ is a field and $p(x)\in F[x]$, then $p(x)$ has at most $n$ roots were $n=$deg($p(x)$)
    
    
    
\noindent\textbf{Problem 2:}
    
    Let $R$ be a principle ideal domain, and let $x\in R$ be nonzero. Prove that $\langle x\rangle$ is maximal if and only if $x$ is irreducible.
    
    
    
\noindent\textbf{Problem 3:}
    
    Suppose that we are given line segments of length $a,b\text{,and }c$. Prove that we can construct solutions to the proportions $\frac{x}{a}=\frac{b}{c}$ and $\frac{a}{y}=\frac{b}{c}$
    
    
    
\noindent\textbf{Additional Problem 1:}
    
    Prove that if $v$ is a Euclidean valuation on $R$ and $x$ and $Y$ are nonzero associates in $R$ then $v(x)=v(y)$.
    
    
    
\noindent\textbf{Additional Problem 2:}
    
    Prove that $\mathbb{Z}[\sqrt{-3}]=\{a+b\sqrt{-3}\mid a,b\in\mathbb{Z}$ is not a unique factorization domain.
    
    
    
\end{document}
