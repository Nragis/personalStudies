\documentclass[a4paper, 11pt]{article}
\usepackage{comment} % enables the use of multi-line comments (\ifx \fi) 
\usepackage{fullpage} % changes the margin
\usepackage[T1]{fontenc}
\usepackage{selinput}
\usepackage{enumitem}
\usepackage{listings}
\usepackage{amsmath}
\usepackage{amssymb}
\usepackage{setspace}

\newcommand{\Mod}[1]{\ (\mathrm{mod}\ #1)}

\begin{document}
%Header-Make sure you update this information!!!!
\noindent
\large\textbf{Modern Abstract Algebra II} \hfill \textbf{Quinn Murphey} \\
\normalsize MAT 4953.001 \hfill Date: 01/30/19 \\
Dr. Patterson \hfill Due Date: 02/01/19 \\
\noindent\makebox[\linewidth]{\rule{\paperwidth}{0.4pt}}

\section*{Problem Set 2}
\doublespacing
\noindent\textbf{Problem 1:}
    
    Let $R$ be a ring, and let $a_1,a_2,\dots ,a_m \in R$. We defined $\langle a_1,a_2,\dots ,a_m\rangle$ to be the set of all elements of $R$ that can be expressed in the form $a_1r_1 + a_2r_2 + \dots a_mr_m$ for some $r_i \in R$. Prove that this is an ideal, and that it is the smallest ideal containing all the elements $a_1,a_2.\dots ,a_m$\\
    
    Assume $a_1.a_2,\dots ,a_m$ are all in $I$ which is an Ideal, so it must absorb multiplication. Therefore, (for all $r\in R$ and for all $a_i)\rightarrow (ra_i \in I$). Therefore since $I$ must also be a subring, it is closed under addition of elements. $r_1a_1, r_2a_2, \dots , r_ma_m$ is in $I$ for all $r_i \in R$. Since our only assumption was that $a_1,a_2,\dots ,a_m$ are all in $I$. $\langle a_1,a_2,\dots ,a_m\rangle$ is the smallest ideal containing all these elements.
    
\noindent\textbf{Problem 2:}
    
    Consider the Ideal $I = \{x^2-x-1\}$ of $\mathbb{Q}[x]$. Prove that fir every natural number $n$, the congruence class $[x^n]$ in $\mathbb{Q}[x]/I$ is equivalent to a congruence class of the form $[a_nx+b_n]$ where $a_n$ and $b_n$ are rational numbers. And find a formula for $a_n,b_n$ in terms of $n$.\\
    
    Because of the definition of division of polynomials, deg($r(x)$)<deg($b(x)$). And deg($b(x)) = 2$. So every congruence class in $\mathbb{Q}[x]/I$ can be represented as $[a_nx+b_n]$ where $a_n$ and $b_n$ are rational numbers. Now we just need to find $a_n$ and $b_n$. We can make a table to help find a pattern\\
    
    \begin{tabular}{ |c|c|c|c|c|c|c|c|c|c| } 
                \hline
                    [$x^n$] & $x^0$ & $x^1$ & $x^2$ & $x^3$ & $x^4$ & $x^5$ & $x^6$ & $x^7$ & $x^8$ \\ 
                \hline
                    $[a_nx+b_n]$ & $1$ & $x$ & $x+1$ & $2x+1$ & $3x+2$ & $5x+3$ & $8x+5$ & $13x+8$ & $21x+13$\\ 
                \hline
            \end{tabular}\\
    
    As we can see, for $n \geq 1$, $a_n = F_n$ and $b_n = F_{n-1}$ where $F_n$ is the $n$th Fibbonacci number. ($F_0 = 0, F_1 = 1)$

\newpage
\noindent\textbf{Problem 3:}
    
    Prove that if $R$ is a ring and $I$ is an ideal of $R$ then $I$ is a prime ideal if and only if $R/I$ is an integral domain.\\
    
    $(\rightarrow )$ Assume $I$ is a prime ideal. That means that if $ab \in I$ implies that $a,b\in I$. For $R/I$ to be an integral domain, there must be two elements $n,m\in R/I$ not in $I$ s.t $nm\in 0$. But since the definition of a prime ideal states that all elements in $I$ also have all of their factors that are in $R$ are also in $I$ so there are no $n,m\in R$ which are not in $I$ s.t $nm=0$.
    
    $(\leftarrow )$ Assume  that $R/I$ is an integral domain. This means there are no elements in $n,m\in R/I$ and not in $I$ s.t $nm\in I$. This means that every single pair $a,b\in R/I$ s.t $ab$ has to be in $I$. This meets the definition of a prime ideal, therefore $I$ is a prime ideal.
    
\noindent\textbf{Problem 4:}
    
    Let $R$ be a ring. Prove that every maximal ideal of $R$ is a prime ideal.\\
    
    Assume $I$ is a maximal ideal of $R$. This means that if you add any element of $R \text{ minus } I$ into $I$ you get $R$. Now assume that $I$ is not a prime ideal. There exists an element $ab\in I$ s.t $a,b\not\in I$. We can use $a$ to get a new ideal generated by $a+I$. Also, since $I$ is maximal, $a+I$ is equal to $R$. Therefore $1\in a+I$. Also: $1=ra+m$ where $r\in R, m\in I$ . Similarly $1=sb+n$ for some $s\in R,n\in I$. Since $$1=1*1=(ra+m)(sb+n)=(rsab+ran+sbm+mn)$$
    and $ab,m,n\in I$ and $I$ absorbs multiplication from $R$, $1\in I$ which means that $I=R$. This contradicts our assumption that $I$ is a maximal ideal, therefore, $I$ is a prime ideal.
    
\noindent\textbf{Problem 14.5:}
    
    Let $S=\{a+bi \mid a,b\in Z, b \text{ is even}$. Show that $S$ is a sub-ring of $Z[i]$, but not an ideal of $Z[i]$.
    
    We need to show that $S$ does not absorb all multiplication from $\mathbb{Z}[i]$. Given an arbitrary element  $n+mi\in\mathbb{Z}[i]$. Multiplied by an element $a+bi\in S$ gives us $$(a+bi)(n+mi)=(an-bm)+(am+bn)i$$
    Since $b$ is even, $bn$ is even, but since $a,m$ can be any integer, their product can be either even or odd. Therefore not every $rs\in S$ where $r\in\mathbb{Z}[i], s\in S$. Therefore S is not an ideal of $\mathbb{Z}[i]$
       
\noindent\textbf{Problem 14.22:}
    
    Let $I= \langle 2\rangle$. Prove that $I[x]$ is not a maximal ideal of $Z[x]$ even though $I$ is a maximal ideal of $Z$
    
    This just requires a counter example. Let $S=\{c_0+c_1x+c_2x^2+\dots +c_nx^n \mid c_i\in\mathbb{Z}, c_0 \text{ is even, and  }n\in\mathbb{N}\} $. We have shown that this is an ideal of $Z[x]$ and its trivial to show that every element of $I$ is in $S$ yet $S$ has more elements such as $2+x$.
    
\noindent\textbf{Problem 14.47:}
    
    Let $R$ be a commutative ring and let $A$ be any subset of $R$. Show that the \textit{annihilator} of $A$, Ann($A$)$=\{r\in R \mid ra=0 $ for all $a\in A\}$, is an ideal.
    
    It is easy to show that Ann($A$) absorbs multiplication if $s*a=0,$ then $rs*a=r*0=0$ $\forall r\in R$. Now we must show that Ann($A$) is a subring of $R$. To do this we must show that it is closed under multiplication and subtraction. \singlespacing
    $$\forall x_1,x_2\in \text{Ann}(A),r\in R((x_1x_2)r=x_1(x_2r)=x_1(0)=0 \rightarrow x_1x_2\in\text{Ann}(A))$$
    $$\forall x_1,x_2\in \text{Ann}(A),r\in R((x_1-x_2)r=x_1r-x_2r=0-0=0\rightarrow x_1-x_\in \text{Ann}(A)$$
    \doublespacing
    Therefore, since Ann($A$) absorbs multiplication and it is a subring of $R$, it is an ideal of $R$
\end{document}
