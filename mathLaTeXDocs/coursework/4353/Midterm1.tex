\documentclass[12pt,letterpaper]{article}
\usepackage{fullpage}
\usepackage[top=2cm, bottom=4.5cm, left=2.5cm, right=2.5cm]{geometry}
\usepackage{amsmath,amsthm,amsfonts,amssymb,amscd}
\usepackage{lastpage}
\usepackage{enumerate}
\usepackage{fancyhdr}
\usepackage{mathrsfs}
\usepackage{xcolor}
\usepackage{graphicx}
\usepackage{listings}
\usepackage{hyperref}
\usepackage{comment}

\hypersetup{
  colorlinks=true,
  linkcolor=blue,
  linkbordercolor={0 0 1}
}
 
\renewcommand\lstlistingname{Algorithm}
\renewcommand\lstlistlistingname{Algorithms}
\def\lstlistingautorefname{Alg.}

\lstdefinestyle{Python}{
    language        = Python,
    frame           = lines, 
    basicstyle      = \footnotesize,
    keywordstyle    = \color{blue},
    stringstyle     = \color{green},
    commentstyle    = \color{red}\ttfamily
}

\setlength{\parindent}{0.0in}
\setlength{\parskip}{0.05in}

% Edit these as appropriate
\newcommand\course{MAT 4353/MAT 5123}
\newcommand\NetID{Pfl955}           % <-- NetID of person #1
\newcommand\StudentName{Quinn Murphey}
\newcommand{\Mod}[1]{\ (\mathrm{mod}\ #1)}

\pagestyle{fancyplain}
\headheight 35pt
\lhead{\course}
\chead{\textbf{\Large Midterm Exam 1}}
\rhead{\StudentName \\ \NetID}
\lfoot{}
\cfoot{}
\rfoot{\small\thepage}
\headsep 1.5em

\begin{document}

\section*{Important Information}
All scratch computation and code will be located in \texttt{exams/midterm1/worksheet.sagews}\\ along with some preliminary write ups of solutions.

\section*{Problem 1}

This problem takes place in the ring ZN $=\mathbb{Z}/N$ where $N=(2^6)^{128}=64^{128}$. Each element in ZN can be represented as a $64$ bit number encoded into a string of characters using UTSASCII. With the leftmost character being the rightmost digit base 64.

The ring UN $=U_N$ has order $\frac{N}{2}$. Let $G=\langle g\rangle$ where $g=$ PublicElementUN. This cyclic subgroup has order $\frac{N}{64}=64^{127}$. Therefore every element $a\in G$ can be represented by$g^c$ where $0\leq c < \frac{N}{64}$. Which can be encoded by a $127$ character string in UTSASCII.

\begin{enumerate}
  \item \textbf{Find the integer form of AliceSecret.}
    
    Using the function StringToInt(AliceSecret) we get a $=30314515205126029548865219310\\ 6339796024903067116313811547587077742056713884059142456100189787735032675453\\ 2748487373363400774963797219063120519209879538661470704510303088855727637394\\ 01798822467525994021.$
    
    \textbf{Find $g^a$}
    
    ZN(StringToInt$(g))^a$
    
    Encoding $g$ back to an integer $< N$ then to an element of ZN, we raise it to the power of $a$ and re-encode it back into UTSASCII to get $g^a=$ ".XnZ0cl3uE.q4\underline{\space\space}v3op5kl2\underline{\space\space}lWYiDP1ds zQujv1Vv.SItl9C.dq35bWQe6SWc.Fy\underline{\space\space}Bh5ffZPtKxqCkdbjzdAj6HhV3ijq2QGHHBCEF 36M6ftoir5JuicZbZ9APH3p2QTl"
    
    (Spaces are replaced with underscores for clarity)
    
  \item \textbf{Given Bob's public key B=BobPublicKey, find the shared key S and encode it in UTSASCII}
    
    From Bob's public key $B=g^b$ for some secret $b$ and Alice's private key $a$ we can generate the shared key $S=g^{ab}=(g^b)^a=B^a$. We get $S=$ "0NIW4I7LWwwJHsWSAgZJe.iqgu6 SVl0IdYZJRhDLMUkFzTLmuwt8wAkJFh\underline{\space\space}v9ovUCf4QBZWABGhev.ioGbs8O2UPke\\ drdEpSxNE6wWvffB7pXqAGvjGO7BFTop61LpqH"
    
    (Spaces are replaced with underscores for clarity)

    
  \item \textbf{Could Alice and Bob conceive of a strategy so their shared secret is secure but not encoded as a string of gibberish? (using D-H)}

    This is the problem of choosing $a$ and $b$ separately based upon $g$ and $g^a$ and $g^b$ are public such that $g^{ab}$ is a desired string. One of the key aspects of D-H is that S must be generated by both parties independently with only $g^a$ and $g^b$ being shared. This means that one of the parties, lets say Alice, must be able to choose an $a$ from Bobs $B = g^b$ such that $B^a=S$ where $S$ is a predecided string in $G$. If we could find a such that $S=(g^b)^a$ for all $g^b$, then we could choose $b=1$ and have an algorithm that could find $a$ such that $S=g^a$ for all $S$. This is the Discrete Logarithm Problem, which if it were solvable, then the Diffie-Hellman Key Exchange would no longer be a viable cryptosystem. So no, there is no way without breaking the Discrete Logarithm Problem that their shared secret is predecided. \cite{Crypto}
\end{enumerate}

\section*{Problem 2}

\begin{itemize}
    \item [1.] \textbf{Write an algorithm for raising an element of S to the power of an integer n}
        
        Using the simple Square and multiply algorithm, we can make an efficient powering algorithm due to the associativity of multiplication. \cite{Crypto}
        \lstset{caption={Raise algorithm}}
        \lstset{label={lst:alg1}}
            \begin{lstlisting}[style = Python]
                def Raise( A, n ):
                binary = bin(n)
                N = [i for i in binary]
                N = N[2:]
                answer = IDTTY
                for i in N:
                    if i == '0':
                        answer = Multiply(answer,answer)
                    if i == '1':
                        answer = Multiply(Multiply(answer,answer),A)
                return(answer)
        \end{lstlisting}
    \item [2.] \textbf{Use the Raise function to raise $a$ to the power of $101^{100}$}
        \begin{align*}
            a &=\text{ "CkKFNe4xQoIPk6he5Mor5OcjqJDJLX2qV4G7VGiABp"}\\
            a^{101^{100}} &= \text{ "i even i would celebrate in rhymes inept. "}  
        \end{align*}
        
    \item [3.] \textbf{Is Fermat's little theorem "true" in the monoid S in the sense that\\ $b^{\text{ORDER}} = b$ for all $b$ in S?}
        
        Raise(a,order) = a $\rightarrow a^\text{ORDER} = a$
        
        It seems that Fermat's little theorem holds true for $a$
        
        However, under the assumption that S is a monoid under multiplications, this implies that not every element a$\in$S necessarily has a multiplicative inverse.
        Rephrasing Fermat's little theorem, we get $\forall b\in S(b*b^{\mid S\mid -1} =$ IDENTITY).
        
        Since S is closed under multiplication, $a^{\mid S\mid -1}\in S$. But Fermat's little theorem implies that $a^{\mid S\mid -1}$ is the multiplicative of $a$ for all $a$, which by definition of monoid is not necessarily true. Therefore Fermat's little theorem is not true in S. But just to be sure that S isn't a group, we will find a counter example.
        
        \begin{align*}
            b&=\text{"1 \underline{\space} \underline{\space} \underline{\space} \underline{\space} \underline{\space} \underline{\space} 0 \underline{\space} \underline{\space} \underline{\space} \underline{\space} \underline{\space} \underline{\space} 0 \underline{\space} \underline{\space} \underline{\space} \underline{\space} \underline{\space} \underline{\space} 0 \underline{\space} \underline{\space} \underline{\space} \underline{\space} \underline{\space} \underline{\space} 0 \underline{\space} \underline{\space} \underline{\space} \underline{\space} \underline{\space} \underline{\space} 0 \underline{\space} \underline{\space} \underline{\space} \underline{\space} \underline{\space} 1"}\\
            b^{64^{42}}&=\text{"1 \underline{\space} \underline{\space} \underline{\space} \underline{\space} \underline{\space} \underline{\space} 0 \underline{\space} \underline{\space} \underline{\space} \underline{\space} \underline{\space} \underline{\space} 0 \underline{\space} \underline{\space} \underline{\space} \underline{\space} \underline{\space} \underline{\space} 0 \underline{\space} \underline{\space} \underline{\space} \underline{\space} \underline{\space} \underline{\space} 0 \underline{\space} \underline{\space} \underline{\space} \underline{\space} \underline{\space} \underline{\space} 0 \underline{\space} \underline{\space} \underline{\space} \underline{\space} \underline{\space} \underline{\space}"}
        \end{align*}
        which is not b. Therefore Fermat's Little Theorem does not hold for all $b\in S$.
        
    \item [4.] \textbf{Assume that the monoid $S$ has a subset $G$ having} `EXPONENT` \textbf{elements such that $G$ is a group under the operation `Multiply`, with identity `IDTTY`.  Prove that $G$ must be equal to the set $S^*$ of *all* invertible elements of $S$.}
        
        Let $q=64$,\\
        $Q=q^6$,\\
       and EXPONENT $=Q(Q-1)(Q-q)(Q-q^2)(Q-q^3)(Q-q^4)(Q-q^5)$
        
        By definition of a group, every element in $G$ is inveritble. Therefore $G \subseteq S^*$. All we must do to prove that $G=S^*$ is prove that $\lvert G\rvert = \lvert S^*\rvert = $ `EXPONENT`. Which will be done below for both Question 3 and the Bonus Question.
        
        First we go into an in depth explanation of the 'Multiply' function.
        
        Given an input of ordered pair $(X,Y)\in S^2$ where $X$ and $Y$ are each $42$ digit $64$-base numbers. Each digit, which is an element of $\mathbb{Z}/64$, is mapped to it's corresponding element in $\mathbb{F}_{64}$ giving us two new elements $A$ and $B$ representable by a 42-tuple with entries in $\mathbb{F}_{64}$.
        \begin{align*}
            \text{Let }A &= 
            \begin{bmatrix}
                a_{0}   & a_{1} & a_{2} & \dots & a_{41} \\
            \end{bmatrix},
            B = 
            \begin{bmatrix}
                b_{0}   & b_{1} & b_{2} & \dots & b_{41} \\
            \end{bmatrix}\\
            &\text{where } a_0,a_1,\dots, a_{41}, b_0,b_1,\dots,b_{41}\in\mathbb{F}_{64}
        \end{align*}
         
        The 'Multiply' function first creates $4$ matrices over the scalar field $\mathbb{F}_{64}$ generated from the elements of $A$ and $B$ as follows:
        \begin{align*}
            MatA &= 
            \begin{bmatrix}
                a_{0}   & a_{1}     & a_{2}     &\dots      & a_{5} \\
                a_{6}   & a_{7}     & a_{8}     &\dots      & a_{11} \\
                a_{12}  & a_{13}    & a_{14}    &\dots      & a_{17} \\
                \vdots  &\vdots     &\vdots     &\ddots     &\vdots \\
                a_{30}  & a_{31}    & a_{32}    &\dots      & a_{35}
            \end{bmatrix}
            && MatB = 
            \begin{bmatrix}
                b_{0}   & b_{1}     & b_{2}     &\dots      & b_{5} \\
                b_{6}   & b_{7}     & b_{8}     &\dots      & b_{11} \\
                b_{12}  & b_{13}    & b_{14}    &\dots      & b_{17} \\
                \vdots  &\vdots     &\vdots     &\ddots     &\vdots \\
                b_{30}  & b_{31}    & b_{32}    &\dots      & b_{35}
            \end{bmatrix}\\
            VecA &=
            \begin{bmatrix}
                a_{36}\\
                a_{37}\\
                a_{38}\\
                \vdots\\
                a_{41}
            \end{bmatrix}
            &&VecB =
            \begin{bmatrix}
                b_{36}\\
                b_{37}\\
                b_{38}\\
                \vdots\\
                b_{41}
            \end{bmatrix}
        \end{align*}
        Then it creates two more matricies $MatC$ and $VecC$ as follows
        \begin{align*}
            MatC &= MatA*MatB  &&VecC=MatA*VecB + VecA
        \end{align*}
        with the elements of $MatC$ and $VecC$ labeled similarly to $MatA$ and $VecA$. Then it returns a 42-tuple $C$ generated from $MatC$ and $VecC$
        \begin{equation*}
            C = 
            \begin{bmatrix}
                c_{0}   & c_{1} & c_{2} & \dots & c_{41}
            \end{bmatrix},
        \end{equation*}
        
        Now we can start analyzing the elements of $S^*$
        
        Let $C=IDDTY$ and $A$ be an arbitrary element in $S$. For $A$ to be invertible, there must exist an element $B$ such that $A*B=C$.
        \begin{equation*}
            MatC = 
            \begin{bmatrix}
                1       & 0         & 0         &\dots      & 0 \\
                0       & 1         & 0         &\dots      & 0 \\
                0       & 0         & 1         &\dots      & 0 \\
                \vdots  &\vdots     &\vdots     &\ddots     &\vdots \\
                0       & 0         & 0         &\dots      & 1
            \end{bmatrix}
            \text{ and }VecC =
            \begin{bmatrix}
                0\\
                0\\
                0\\
                \vdots\\
                0
            \end{bmatrix}
        \end{equation*}
        We can see here that $MatC$ is equal to I$_{6,6}$ and $VecC$ is equal to the $6\times 1$ zero vector.
        We can rewrite the multiplication equations below
        \begin{align*}
            MatC &= MatA*MatB  &&VecC=MatA*VecB + VecA\\
            MatB &= (MatA)^{-1}*MatC &&VecB=(MatA)^{-1}(VecC-VecA)\\
            MatB &= (MatA)^{-1} &&VecB=(MatA)^{-1}(-VecA)
        \end{align*}
        Now we have a formula for $B$ based only upon $A$. However, $B$ only exists when $MatA$ is nonsingular.
        
        Therefore, $A$ is invertible under 'Multiply' if and only if $MatA$ is nonsingular. Since if $A$ is invertible, it's easy to show $MatB=(MatA)^{-1}$.
        
        Since we are trying to find the order of $S^*$, we can use the fact that for an arbitrary $a\in S$, (where $P(A)$ denotes "probability of A")
        $$\frac{\lvert S*\rvert}{\lvert S\rvert}=P(a\in S^*)$$
        $$\Rightarrow \lvert S*\rvert = P(a\in S^*)*\lvert S\rvert$$
        \text{And we know } $\lvert S\rvert = 64^{42}$ so we have
        \begin{equation}
            \Rightarrow \lvert S*\rvert = P(a\in S^*)*64^{42}    
        \end{equation}
        \newpage
        $MatA$ is nonsingular if and only if $\det(MatA) \not = 0$ and $\det(MatA) \not = 0$ if and only if the rows of $MatA$ are linearly independent. We can find the probability that this is true given random entries from $\mathbb{F}_{64}$. (We will use LI to mean Linearly Independent from now on). Using the probability of unions of events formula. \cite{Probability}
        \begin{align*}
            &P(MatA\text{ is nonsingular}) = P(\text{All 6 rows are LI})=\\ &P(\text{The 1st row is LI from those above it})*\\ *&P(\text{The 2nd row is LI from those above it given that the first row is LI})*\dots*\\ *&P(\text{The 6th row is LI from those above it given the first 5 rows are LI})
        \end{align*}
        
        Each row has $6$ entries and each entry has $64$ options from $\mathbb{F}_{64}$, so each row has $64^6$ unique possibilities. Additionally, each row has $64$ unique possible scalar multiples: $\mathbb{F}_{64}(0)$ through $\mathbb{F}_{64}(63)$ 
        
        The first row is trivially linearly independent from the empty set of rows above it only when the row is nonzero which is all but 1 of $64^{6}$ possibilities so it has probability $$(1-\frac{1}{64^6})=(1-64^{-6})$$\\
        The second row: there are $64$ out of $64^6$ possibilities which are scalar multiples of the first row so the probability that it is linearly independent $$(1-\frac{64}{64^6})=(1-64^{-5})$$
        The third row: there are $64^2$ linear combinations of row 1 and 2 which are all unique \cite{Apostols}. Therefore the probability that row three is linearly independent from rows 1 and 2 is $$(1-\frac{64^2}{64^6})=(1-64^{-4})$$
        The fourth row: there are $64^3$ unique linear combinations of rows 1, 2, and 3. Therefore the probability that row four is linearly independent from rows 1, 2, and 3 is $$(1-\frac{64^3}{64^6})=(1-64^{-3})$$
        We can proceed repeatedly due to unique decomposition into linearly independent components. And the fact that each probability we assume that the above rows are linearly independent.
        \newpage
        We obtain the following probability 
        \begin{equation}
            P(\text{All 6 rows are LI})=(1-\frac{1}{64^6})(1-\frac{64}{64^6})(1-\frac{64^2}{64^6})(1-\frac{64^3}{64^6})(1-\frac{64^4}{64^6})(1-\frac{64^5}{64^6})    
        \end{equation}
        
        From the equation (1) and (2) we obtain 
        \begin{equation*}
            \lvert S^*\rvert = (64^{42})*(1-\frac{1}{64^6})(1-\frac{64}{64^6})(1-\frac{64^2}{64^6})(1-\frac{64^3}{64^6})(1-\frac{64^4}{64^6})(1-\frac{64^5}{64^6})
        \end{equation*}
        If we let $q=64$ and $Q=64^6$ we can distribute the $64^{42}$ evenly throughout the 7 terms to obtain
        \begin{equation}
            \lvert S^*\rvert = Q(Q-1)(Q-q)(Q-q^2)(Q-q^3)(Q-q^4)(Q-q^5)
        \end{equation}
        
        Which we can immediately recognize as the same value as EXPONENT'. $S^*$ and $G$ are both finite and $G\subseteq S^*$ as shown above. We assumed that $\lvert G\rvert=$ EXPONENT', and now we have $\lvert S^*\rvert=$ EXPONENT' as well. Therefore $\lvert S^*\rvert = \lvert G\rvert$ and $G=S^*$ finishing problem 3. \cite{Algebra}
        
        Note: I went for a far more in depth proof than necessary for problem 3 in order to make the bonus question trivial.
        
    \item [Bonus.] \textbf{Prove when the formula }$a^\text{EXPONENT}= \text{IDENTITY}$ \textbf{ holds in S}
    
        Let $H=\{a\in S \mid a^\text{EXPONENT}= \text{IDENTITY}\}$. First we will show that $H$ is equal to $S^*$. 
        
        Let $h\in H$. Then $h^\text{EXPONENT}= \text{IDENTITY}$ and $h*h^\text{EXPONENT-1}= \text{IDENTITY}$. This means that $h$ has an inverse and therefore $h\in S^*$ and $H\subseteq S^*$.
        
        Since $\lvert S^* \rvert = $ 'EXPONENT' (3), we can apply Lagrange's Theorem \cite{Algebra} to every element in $S^*$ and find that for all $s\in S^*$, $s^\text{EXPONENT}=\text{IDENTITY}$. Which gives us $S^*\subseteq H$
        
        Combining these two statements we get $H=S^*$. Since $H$ was the set of all $a\in S$ such that $a^\text{EXPONENT}= \text{IDENTITY}$. We have shown that $a\in S^*$ if and only if $s^\text{EXPONENT}= \text{IDENTITY}$. Concluding our proof.
        
        The formula $A^\text{EXPONENT}= \text{IDENTITY}$ holds true in $S$ if and only if $A$ is invertible which is the case if and only if $MatA$ (same as defined earlier) is nonsingular.
     
        \newpage
    \item [Easter Egg.] \textbf{Given the 23rd power of an element, find that element}
        
        EasterEgg23 = Raise(EasterEgg, 23) = \\"DONumAafOKzQkUC7w1VG33uIJKQOPEMYfDFbpGlO47"
        
        First thing I did was verify that $EasterEgg23\in S^*$ by raising it to the power of 'EXPONENT' which it was. Then I needed to find a way to get $A$ from $A^{23}$. So I set up the equation 
        \begin{align*}
            (EasterEgg23)^u&=A^1\\
            (A^{23})^u&=A^1\\
            (A^\text{EXPONENT})^u*A^{23u} &= (A^\text{EXPONENT})^z*A^1\\
            A^{y*\text{EXPONENT}+23u} &= A^{z*\text{EXPONENT} + 1}
        \end{align*}
        So, we can write:
        \begin{align*}
            23u + \text{EXPONENT}*v &= 1
        \end{align*}
        which is of the same form as the Extended Euclid Algorithm \cite{Crypto} since\\ $$\text{gcd}(23,\text{EXPONENT})=1$$
        We can use the algorithm below to find the values of $u$ and $v$. 
        \lstset{caption={Extended Euclidean Algorithm}}
        \lstset{label={1st:alg1}}
            \begin{lstlisting}[style = Python]
                a=23
                b=EXPONENT
                u=1
                g=a
                y=b
                x=0
                while y!=0:
                    t= g%y
                    q= (g-t)/y
                    s= u-q*x
                    u=x
                    g=y
                    x=s
                    y=t
                v=(g-a*u)/b
                print("v: ",v)
                print("u: ",u)
                print("g: ",g)
        \end{lstlisting}
         
        From this we find
        \begin{align*}
            u &= 3096591532549975869713630575798007667064299756126252922045308095519101106087\\
            v &= -10
        \end{align*}
        We can then write that 
        \begin{align*}
            23*u &\equiv 1 \Mod{\text{EXPONENT}}\\
            (A^{23})^u&=A^1 && (\text{Since $A^\text{EXPONENT}=\text{IDENTITY}$})\\
            (\text{EasterEgg23})^u&=A
        \end{align*}
        From this we get the string: $$(\text{EasterEgg23})^u = A = \text{"You have found Easter Egg 2 worth 3 points"}$$
\end{itemize}

\section*{Problem 3:} 

    All functions used in this section can be found in \texttt{exams/midterm1/Question3.sage} which are used in the Question 3 section of \texttt{exams/midterm1/worksheet.sagews}
    \begin{equation*}
        \text{Let } A = 
        \begin{bmatrix}
            1   & 1     & 1                         & 1                     & 1\\
            1   & W     & W^2                       & W^3                   & W^4\\
            1   & W^2   & W^4                       & W^4 + W^3 + W + 1     & W^5 + W^4 + W^2 + W + 1\\
            1   & W^3   & W^4 + W^3 + W + 1         & W^5 + W^4 + W^2 + 1   & W^5 + W^3 + 1\\
            1   & W^4   & W^5 + W^4 + W^2  + W + 1  & W^5 + W^3 + 1         & W^4 + W + 1
        \end{bmatrix}
    \end{equation*}
    where each element in $A$ is an element of $\mathbb{F}_{64}$
\begin{itemize}
    \item [1.] \textbf{Using the given cipher encrypt AlicesMessage}
    \begin{align*}
        \text{AlicesMessage} &= \text{"When and where are we meeting."}\\
        \text{EncryptedMessage} &= \text{"ZdMhgXB2sFd3qlTUF2B5nRRBeWx7GR"}
    \end{align*}
        You obtain this by first separating AlicesMessage into 5 character chunks, then turning each chunk into a 5x1 matrix, then multiplying it by $A$. After this, you stitch the encrypted chunks back together in the same order to get your EncryptedMessage.
    \item [2.] \textbf{Decrypt Bob's secret message}
        \begin{align*}
            \text{BobsSecret} &= \text{"Kdkk93R.N4eyrEc20XNFktfgNyrpuD"}\\
        \text{DecryptedMessage} &= \text{"Meet at dusk under tall tree. "}
        \end{align*}
        First you must find $A^{-1}$ which is computed using Sage which is far too big to display here. Then you must run the encryption algorithm except using $A^{-1}$ instead of $A$
         
        
    \item [3.] \textbf{Given a corresponding string of plaintext and ciphertext find the encryption matrix A}
        \begin{align*}
            \text{EncryptedMessage} &= \text{"tjcp0Yyd aC.liOCb65.5liol"}\\
            \text{PlaintextMessage} &= \text{"Is Evil Eve listening in."}
        \end{align*}
        This is possible because we have enough data to populate a $5\times 5$ corresponding plaintext and ciphertext array. We can convert each chunk of both ciphertext and plain text into a $5\times 1$ matrix, labelling our cipherChunks $c_0,c_1,c_2,c_3,c_4$ and our plainChunks $p_0,p_1,p_2,p_3,p_4$ relative to each other. We can then create two matrices:
        \begin{align*}
            C =
            \begin{bmatrix}
                c_0 & c_1 & c_2 & c_3 & c_4
            \end{bmatrix}
            && X = 
            \begin{bmatrix}
                x_0 & x_1 & x_2 & x_3 & x_4
            \end{bmatrix}
        \end{align*}
        with the property \cite{Apostols,Crypto}
        \begin{align}
            C = AX
        \end{align}
        where $A$ is our encryption matrix.
        
        Once we do this, it is easy to find A from (4) by writing
        \begin{equation}
            C(X^{-1})=A
        \end{equation}
        Luckily for us $X$ is nonsingular. Meaning we can find $X^{-1}$. Plugging $C$ and $X^{-1}$ into equation (5) we get
\end{itemize}
        \begin{equation*}
            A = 
            \begin{bmatrix}
                (1,0,0,1,0,1)       &(0,0,1,1,1,0)          &(0,0,0,0,0,0)      &(0,1,1,0,1,0)      &(1,1,1,1,1,0)\\
                (1,1,0,1,0,0)       &(1,0,0,1,0,1)          &(1,1,0,0,1,0)      &(1,1,0,1,0,0)      &(0,0,1,1,1,0)\\
                (1,0,1,1,1,0)       &(1,0,0,0,1,0)          &(1,0,1,1,1,0)      &(0,0,0,1,1,0)      &(1,1,0,0,1,0)\\
                (1,0,1,1,1,0)       &(1,0,0,0,1,0)          &(0,0,0,0,0,0)      &(0,0,0,0,0,0)      &(0,0,0,1,1,0)\\
                (1,1,1,1,0,0)       &(0,0,1,0,0,0)          &(1,0,1,0,0,1)      &(0,0,0,1,1,1)      &(1,0,0,0,1,0)
            \end{bmatrix}
        \end{equation*}
\begin{itemize}
    \item[ ]with each each element $a+bW+cW^2+dW^3+eW^4+fW^5$ being represented as the ordered 6-tuple $(a,b,c,d,e,f)$.
        This is done in order for this to fit on the LaTeX document, to see it in normal form, see Question 3 Part 3 of \texttt{worksheet.sagews}
        
        Then, to test whether $A$ is the correct matrix, we decrypt EncryptedMessage using our found $A$, and we find FullDecrypt(EncryptedMessage, A) = PlaintextMessage which proves $A$ is our encryption matrix.
         
        
    \item [Bonus.] \textbf{Given a ciphertext and plaintext pair and an affine cypher, can you determine the key $(A,b)$}
    
    We are trying to find a $5\times 5$ matrix $A$ and a $5\times 1$ matrix $b$ from 5 chunks of both plaintext and corresponding cyphertext
    
    Like above, we can create two matrices 
    \begin{align*}
        C =
        \begin{bmatrix}
            c_0 & c_1 & c_2 & c_3 & c_4
        \end{bmatrix}
        && X = 
        \begin{bmatrix}
            x_0 & x_1 & x_2 & x_3 & x_4
        \end{bmatrix}
    \end{align*}
    And we have the equation:
    \begin{align*}
        C = AX+B\\
        \text{Where }B=
        \begin{bmatrix}
            b   &b  &b  &b  &b
        \end{bmatrix}
    \end{align*}
    Therefore we have 1 equation and two unknowns $A$ and $b$ which is unsolvable.
    
    Therefore, with only $5$ chunks of plaintext and cyphertext we cannot find both $A$ and $b$. 
    
    However, if we were given 6 chunks we could set up the two equations 
    \begin{align*}
        C&=AX+B\\
        c_5&=Ax_5+b
    \end{align*}
    And we have a linear system of equations with two unknowns which is easily solvable.
\end{itemize}
    
\section*{Problem 4:}
An \textit{affine} n-cypher is a function from $\mathbb{R}^n$ to $\mathbb{R}^n$ of the form $f_{A,b}(x) = Ax + b$ where $A\in GL_n(\mathbb{R})$ and $b\in \mathbb{R}^n$ are fixed.
\begin{enumerate}
    \item \textbf{Prove that affine n-cyphers over $\mathbb{R}$ form a group under composition}
    
    We will deonte this group $\mathcal{A}_n$. \cite{Algebra}
    \begin{itemize}
        \item [Associativity:] Function Composition is Associative
        \item [Identity:] Let $A$ be the $n\times n$ identity matrix and $b$ be the $n\times 1$ zero vector. Then $f_{A,b}(x)=I_{n}x + 0= x$ for all $x$, which is the identity function. Since $I_n$ is invertible, $I_n\in GL_n(\mathbb{R})$ so $f_{A,b}\in\mathcal{A}_n$. Therefore the identity function is in $\mathcal{A}_n$
        \item [Inverse:] Let $f_{A,b}(x)=Ax+b$ be an arbitrary element of $\mathcal{A}_n$. Let $C=A^{-1}$ and $d= -A^{-1}b$. Then the function $f_{C,d}(x)=A^{-1}x-A^{-1}b$ is in $\mathcal{A}_n$ because $A$ is invertible and $A^{-1}b\in \mathbb{R}^n$. This function is also the inverse of $f_{A,b}$ because
        \begin{align*}
            f_{C,d}(f_{A,b}(x)) &= f_{C,d}(Ax+b)\\
            &= C(Ax+b)+d\\
            &= CAx+Cb+d\\
            &= (A^{-1})(A)x+(A^{-1})b - (A^{-1})b\\
            &= I_nx\\
            &= x
        \end{align*}
        where $I_n$ is the $n\times n$ identity matrix. This proves that every element of $\mathcal{A}_n$ has an inverse.
        \newpage
        \item [Closure:] Let $A,C\in GL_n(\mathbb{R})$ and $b,d\in \mathbb{R}^n$. Then for any two functions $f_{A,b}$ and $f_{C,d}$ their composition $f_{A,b}\circ f_{C,d}$ is the mapping from $x \mapsto f_{A,b}(f_{C,d}(x)).$ 
        \begin{align*}
            f_{A,b}(f_{C,d}(x)) &= f_{A,b}(Cx+d)\\
            &= A(Cx + d) + b\\ 
            &= ACx + Ad + b
        \end{align*}
        Since $GL_n$ is closed under multiplication (due to shoes and socks) $AC\in GL_n(\mathbb{R})$. Also both $Ad$ and $b$ are in $\mathbb{R}^n$ so their sum also is since $\mathbb{R}^n$ is a group under addition. Therefore for $Y=AC$ and $z=Ad+b$, $f_{Y,z}\in\mathcal{A}_n$. Therefore the composition of two affine n-cypers is an affine n-cipher
         
    \end{itemize}
    \item \textbf{Prove the set of transformations $F = \{f_{A,0} \mid A\in GL_n(\mathbb{R})\}$ is a subgroup of $\mathcal{A}_n$ and isomorphic to $GL_n(\mathbb{R})$}
    
    We must prove that $F$ is nonempty, closed under composition, and every element is invertible. $F$ is obviously nonempty since $GL_n(\mathbb{R})$ is nonempty and $f_{A,0}$ is defined for all $A\in GL_n(\mathbb{R})$. \cite{Algebra}
    \begin{itemize}
        \item [Closure:] Let $f_{A,0},f_{C,0}\in F$, then their composition: 
        \begin{align*}
            (f_{A,0}\circ f_{C,0})(x) &= f_{A,0}(f_{C,0}(x))\\
            &= f_{A,0}(Cx)\\
            &= ACx\\
            &= f_{AC,0}(x)
        \end{align*}
        is also in $F$ because $AC\in GL_n(\mathbb{R})$ since $GL_n(\mathbb{R})$ is a group under multiplication and $b$ still equals 0.
        
        \item [Inverse:] As stated above the inverse of any $f_{A,b}$ is $f_{A^{-1},-A^{-1}b}$. Since $b=0$ this simplifies to $f_{A^{-1},0}$ and if $A\in GL_n(\mathbb{R})$ then $A^{-1}\in GL_n(\mathbb{R})$ by definition of the General Linear Group. So, $f_{A^{-1},0}\in F$. Therefore $(f_{A,0})^{-1}\in F$ for all $f_{A,0}\in F$.
    \end{itemize}
    
    Therefore, $F$ is a subgroup of $\mathcal{A}_n$.
    
    Define a function $\phi: F \rightarrow GL_n(\mathbb{R})$ such that $f_{A,0} \mapsto A\in GL_n(\mathbb{R})$. We know this exists for all $f_{A,0}\in F$ because $A\in GL_n(\mathbb{R})$ by definition. 
    
    Now to prove that $\phi$ is a homomorphism. \cite{Algebra} Let $f_{A,0},f_{C,0}\in F$
    \begin{align*}
        \phi(f_{A,0}\circ f_{C,0}) &= \phi(f_{A,0}(Cx))\\
        &= \phi(ACx)\\
        &= AC\\
        &= A*C\\
        &= \phi(Ax)*\phi(Cx)\\
        &= \phi(f_{A,0})*\phi(f_{C,0})
    \end{align*}
    This function always exists since $GL_n(\mathbb{R})$ is a group under multiplication so $AC\in GL_n(\mathbb{R})$. Therefore $\phi$ is a homomorphism from $F$ to $GL_n(\mathbb{R})$ and from composition to multiplication.
    
    Now to prove this homomorphism is bijective:
    \begin{itemize}
        \item [Injective:] Let $f_{A,0}=f_{C,0}\in F$ then $Ax=Cx$ for all $x\in \mathbb{R}^n$ therefore $A=C$ since $A,C$ are invertible by definition. Therefore 
        \begin{align*}
            \phi(f_{A,0}) &= A\\
            &= C\\
            &= \phi(f_{C,0})
        \end{align*}
        for any two equal elements in $F$. Therefore $\phi$ is injective.
        
        \item [Surjective:] Let $f_{A,0},f_{C,0}\in F$ such that $\phi(f_{A,0})=\phi(f_{C,0})$. Then, $A=C$. Then, $Ax=Cx$ for all $x\in \mathbb{R}^n$. Then by definition, $f_{A,0}=f_{C,0}$. Therefore $\phi$ is surjective.
    \end{itemize}
    
    We have proven both $\phi$ is a homomorphism, and that $\phi$'s bijective. Therefore, $\phi$ is an isomorphism and $F$ is isomorphic to $GL_n(\mathbb{R})$ \cite{Algebra}
    
    \textbf{Prove the set of transformations $H = \{f_{I_n,b} \mid b\in \mathbb{R}^n\}$ is a subgroup of $\mathcal{A}_n$ and isomorphic to $\mathbb{R}^n$}
    
    $I_n = $ the $n\times n$ identity matrix.
    
    We must prove that $H$ is nonempty, closed under composition, and every element is invertible. $H$ is obviously nonempty since $\mathbb{R}^n$ is nonempty and $f_{I_n,b}$ is defined for all $b\in \mathbb{R}^n$. 
    \begin{itemize}
        \item [Closure:] Let $f_{I_n,b},f_{I_n,d}\in H$, then their composition:
        \begin{align*}
            (f_{I_n,b}\circ f_{I_n,d})(x) &= f_{I_n,b}(f_{I_n,d}(x))\\
            &= f_{I_n,b}(I_nx+d)\\
            &= I_n(I_nx+d)+b\\
            &= I_nx + (d+b)\\
            &= f_{I_n,d+b}(x)
        \end{align*}
        is also in $H$ because $d+b\in \mathbb{R}^n$ since $\mathbb{R}^n$ is closed under addition and $A$ still equals $I_n$
        \item [Inverse:] As stated above, the inverse of any $f_{A,b}$ is $f_{A^{-1},-A^{-1}b}$. Since $A=I_n$ and $A^{-1}=I_n$ this simplifies to $f_{I_n,-I_nb}=f_{I_n,-b}$ and $-b\in\mathbb{R}^n$ because $\mathbb{R}^n$ is a group under addition. So $f_{I_n,-b}\in H$ and therefore $(f_{I_n,b})^{-1}\in H$ for all $f_{I_n,b}\in H$
    \end{itemize}
    Therefore, $H$ is a subgroup of $\mathcal{A}_n$.
    
    Define a function $\phi: H \rightarrow \mathbb{R}^n$ such that $f_{I_n,b} \mapsto b\in \mathbb{R}^n$. We know this exists for all $f_{I_n,b}\in H$ because $b\in\mathbb{R}^n$ by definition. 
    
    Now to prove that $\phi$ is a homomorphism. Let $f_{I_n,b},f_{I_n,d}\in H$
    \begin{align*}
        \phi(f_{I_n,b}\circ f_{I_n,d}) &= \phi(f_{I_n,b}(I_nx+d))\\
        &= \phi(I_nI_nx + I_nd + b)\\
        &= \phi(x+(d+b))\\
        &= (d+b)\\
        &= b+d\\
        &= \phi(I_nx+b)+\phi(I_nx+d)\\
        &= \phi(f_{I_n,b})+\phi(f_{I_n,d})
    \end{align*}
    This function always exists since $\mathbb{R}^n$ is an abelian group under addition so $d+b=b+d\in \mathbb{R}^n$. Therefore $\phi$ is a homomorphism from $H$ to $\mathbb{R}^n$ and from composition to addition.
    
    Now to prove this homomorphism is bijective:
    \begin{itemize}
        \item [Injective:] Let $f_{I_n,b}=f_{I_n,d}\in H$, then $I_nx+b=I_nx + d$ for all $x\in\mathbb{R}^n$. We can subtract $x$ from both sides and obtain $b=d$. Therefore
        \begin{align*}
            \phi(f_{I_n,b}) &= b\\
            &= d\\
            &= \phi(f_{I_n,d})
        \end{align*}
        for any two equal elements in $H$. Therefore $\phi$ is injective
        \item [Surjective:] Let $f_{I_n,b},f_{I_n,d}\in H$ such that $\phi(f_{I_n,b})=\phi(f_{I_n,d})$. Then, $b=d$ and $I_nx+b=I_nx+d$ (add $I_nx$ to both sides) for all $x\in\mathbb{R}^n$. Then, by definition $f_{I_n,b}=f_{I_n,d}$. Therefore $\phi$ is surjective.
    \end{itemize}
    We have proven both that $\phi$ is a homomorphism, and that $\phi$'s bijective. Therefore $\phi$ is an isomorphism and $H$ is isomorphic to $\mathbb{R}^n$.
    
\end{enumerate}
\newpage

\section*{Additional Bonus Problem}

\textbf{Why does $W^{1000000}=W$?}

$W*1=W$ and $a^{63}=1$ for all $a\in\mathbb{F}_{64}$ due to Lagrange's Theorem since $\mathbb{F}_{64}^*$ is a multiplicative group with order 63 \cite{Algebra}. This is because $\mathbb{F}_{64}^*=\mathbb{F}_{64}/\{0\}$. Therefore 
\begin{align*}
    W^{63}&=1\\
    (W^{63})^{15873}&=1\\
    W^{999999}&=1\\
    W*W^{999999}&=W\\
    W^{1000000}=W
\end{align*}
This is why $W^{1000000}=W$ in $\mathbb{F}_{64}$. Note: Powers add because multiplication in $\mathbb{F}_{64}$ is associative.

\begin{thebibliography}{9}
\bibitem{Apostols} 
Tom M. Apostol. 
\textit{Apostol's Calculus Volume 2}. 
 
\bibitem{Crypto} 
Jeffrey Hoffstein, Jill Pipher, Joseph H. Silverman.
\textit{An Introduction to Mathematical Cryptography}
 
\bibitem{Algebra} 
Joseph A. Gallian
\textit{Contemporary Abstract Algebra}

\bibitem{Probability}
Kai Lai Chung, Farid AitSahlia
\textit{Elementary Probability Theory}

\end{thebibliography}

\end{document}