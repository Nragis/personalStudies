\documentclass[a4paper, 11pt]{article}
\usepackage{comment} % enables the use of multi-line comments (\ifx \fi) 
\usepackage{fullpage} % changes the margin
\usepackage[T1]{fontenc}
\usepackage{selinput}
\usepackage{enumitem}
\usepackage{listings}
\usepackage{amsmath}
\usepackage{amssymb}
\usepackage{setspace}

\SelectInputMappings{
   aacute={á},
   ntilde={ñ}
}
\newcommand{\Mod}[1]{\ (\mathrm{mod}\ #1)}

\begin{document}
%Header-Make sure you update this information!!!!
\noindent
\large\textbf{Math Fdn. of Cryptography} \hfill \textbf{Quinn Murphey} \\
\normalsize MAT 4353.001 \hfill Date: 01/23/19 \\
Dr. Dueñez \hfill Due Date: 01/29/19 \\
\noindent\makebox[\linewidth]{\rule{\paperwidth}{0.4pt}}
\section*{Week 2}
1.15, 1.19, 1.20, 1.25, 1.26(a) (show step-by-step, possibly aided by a calculator or computer), 1.27, 1.33, 1.34(d),(e),(f),(g), 1.36(a)\&(d).  Extra credit: 1.37, 1.38.
\section*{Mandatory}
\noindent\textbf{1.15}
    \begin{enumerate}[label=(\alph*)]
        \item Make addition and multiplication tables for $(\mathbb{Z}/3\mathbb{Z})$\\
            \begin{tabular}{ |c|c|c|c| } 
                \hline
                    + & 0 & 1 & 2\\ 
                \hline
                    0 & 0 & 1 & 2\\ 
                \hline
                    1 & 1 & 2 & 0\\
                \hline
                    2 & 2 & 0 & 1\\
                \hline
            \end{tabular}
            \quad
            \begin{tabular}{ |c|c|c|c| } 
                \hline
                    $\times$ & 0 & 1 & 2\\ 
                \hline
                    0 & 0 & 0 & 0\\ 
                \hline
                    1 & 0 & 1 & 2\\
                \hline
                    2 & 0 & 2 & 1\\
                \hline
            \end{tabular}
        \item Make addition and multiplication tables for $(\mathbb{Z}/6\mathbb{Z})$\\
            \begin{tabular}{ |c|c|c|c|c|c|c| } 
                \hline
                    + & 0 & 1 & 2 & 3 & 4 & 5\\ 
                \hline
                    0 & 0 & 1 & 2 & 3 & 4 & 5\\ 
                \hline
                    1 & 1 & 2 & 3 & 4 & 5 & 0\\ 
                \hline
                    2 & 2 & 3 & 4 & 5 & 0 & 1\\ 
                \hline
                    3 & 3 & 4 & 5 & 0 & 1 & 2\\ 
                \hline
                    4 & 4 & 5 & 0 & 1 & 2 & 3\\  
                \hline
                    5 & 5 & 0 & 1 & 2 & 3 & 4\\ 
                \hline
            \end{tabular}
            \quad
            \begin{tabular}{ |c|c|c|c|c|c|c| } 
                \hline
                    $\times$ & 0 & 1 & 2 & 3 & 4 & 5\\ 
                \hline
                    0 & 0 & 0 & 0 & 0 & 0 & 0\\ 
                \hline
                    1 & 0 & 1 & 2 & 3 & 4 & 5\\ 
                \hline
                    2 & 0 & 2 & 4 & 1 & 3 & 5\\ 
                \hline
                    3 & 0 & 3 & 1 & 3 & 1 & 3\\ 
                \hline
                    4 & 0 & 4 & 2 & 2 & 2 & 2\\ 
                \hline
                    5 & 0 & 5 & 1 & 5 & 1 & 5\\ 
                \hline
            \end{tabular}
        \item Make addition and multiplication tables for $(\mathbb{Z}/9\mathbb{Z})^*$\\
            \begin{tabular}{ |c|c|c|c|c|c|c|c| } 
                \hline
                    + & 1 & 2 & 4 & 5 & 7 & 8\\ 
                \hline
                    1 & 2 & 3 & 5 & 6 & 8 & 0\\ 
                \hline
                    2 & 3 & 4 & 6 & 7 & 0 & 1\\ 
                \hline
                    4 & 5 & 6 & 8 & 0 & 2 & 3\\ 
                \hline
                    5 & 6 & 7 & 0 & 1 & 3 & 4\\ 
                \hline
                    7 & 8 & 0 & 2 & 3 & 5 & 6\\ 
                \hline
                    8 & 0 & 1 & 3 & 4 & 6 & 7\\ 
                \hline
            \end{tabular}
            \quad
            \begin{tabular}{ |c|c|c|c|c|c|c|c| } 
                \hline
                    $\times$ & 1 & 2 & 4 & 5 & 7 & 8\\ 
                \hline
                    1 & 1 & 2 & 4 & 5 & 7 & 8\\ 
                \hline
                    2 & 2 & 4 & 8 & 1 & 5 & 7\\ 
                \hline
                    4 & 4 & 8 & 7 & 2 & 1 & 5\\ 
                \hline
                    5 & 5 & 1 & 2 & 7 & 8 & 4\\ 
                \hline
                    7 & 7 & 5 & 1 & 8 & 4 & 2\\ 
                \hline
                    8 & 8 & 7 & 5 & 4 & 2 & 1\\ 
                \hline
            \end{tabular}
        \item Make addition and multiplication tables for $(\mathbb{Z}/16\mathbb{Z})^*$\\
            \begin{tabular}{ |c|c|c|c|c|c|c|c|c| } 
                \hline
                    + & 1 & 3 & 5 & 7 & 9 & 11 & 13 & 15\\ 
                \hline
                    1 & 2 & 4 & 6 & 8 & 10 & 13 & 14 & 0\\ 
                \hline
                    3 & 4 & 6 & 8 & 10 & 12 & 14 & 0 & 2\\ 
                \hline
                    5 & 6 & 8 & 10 & 12 & 14 & 0 & 2 & 4\\ 
                \hline
                    7 & 8 & 10 & 12 & 14 & 0 & 2 & 4 & 6\\ 
                \hline
                    9 & 10 & 12 & 14 & 0 & 2 & 4 & 6 & 8\\ 
                \hline
                    11 & 12 & 14 & 0 & 2 & 4 & 6 & 8 &10\\ 
                \hline
                    13 & 14 & 0 & 2 & 4 & 6 & 8 & 10 & 12\\ 
                \hline
                    15 & 0 & 2 & 4 & 6 & 8 & 10 & 12 & 14\\ 
                \hline
            \end{tabular}
            \quad
            \begin{tabular}{ |c|c|c|c|c|c|c|c|c| } 
                \hline
                    $\times$ & 1 & 3 & 5 & 7 & 9 & 11 & 13 & 15\\ 
                \hline
                    1 & 1 & 3 & 5 & 7 & 9 & 11 & 13 & 15\\ 
                \hline
                    3 & 3 & 9 & 15 & 5 & 11 & 1 & 7 & 13\\ 
                \hline
                    5 & 5 & 15 & 9 & 3 & 13 & 7 & 1 & 11\\ 
                \hline
                    7 & 7 & 5 & 3 & 1 & 15 & 13 & 11 & 9\\ 
                \hline
                    9 & 9 & 11 & 13 & 15 & 1 & 3 & 5 & 7\\ 
                \hline
                    11 & 11 & 1 & 7 & 13 & 3 & 9 & 15 & 5\\ 
                \hline
                    13 & 13 & 7 & 1 & 11 & 5 & 15 & 9 & 3\\ 
                \hline
                    15 & 15 & 13 & 11 & 9 & 7 & 5 & 3 & 1\\ 
                \hline
            \end{tabular}
    \end{enumerate}
\doublespacing
\noindent\textbf{1.19}
    Prove that if $a_1$ and $a_2$ are units modulo $m$, then $a_1a_2$ is a unit modulo $m$\\
    
    Let $G$ be a multiplicative group and $a_1,a_2\in R$ are units(as are all elements of G). Let $b_1$ and $b_2$ be multiplicative inverses for $a_1$ and $a_2$ respectively. We know that both $b_1$ and $b_2$ are elements of R because $a_1$ and $a_2$ are units. Now we can prove that $b_2b_1$ is the inverse of $a_1a_2$ on both sides.
    \singlespacing
    $$(a_1a_2)(b_2b_1)$$
    $$= a_1(a_2b_2)b_1$$
    $$= a_1b_1$$
    $$= 1$$
    $$= b_2a_2$$
    $$= b_2(b_1a_1)a_2$$
    $$= (b_2b_1)(a_1a_2)$$
    \doublespacing
    Since $G$ a generalization of $U_m$, this holds true for all units mod m\\
    
\noindent\textbf{1.20}
    Prove that $m$ is prime if and only if $\phi\left(m\right) = m-1$\\
    
    We will first prove $m$ is prime $\rightarrow$ $\phi\left(m\right) = m-1$. 
    First, define $\phi (m) := #U_m$. We can use the definition of prime to factor out $m$ into it's prime factors: $m$. Then we can use the Fundamental Theorem of Arithmetic for each number $a$ less than $m$ and greater than $0$ to factor each $a$ into it's prime factors $a_1,a_2,\dots ,a_n (n>0)$. Since each $a_i \leq a$, and $a < m$, no factor of $a$ is equal to the only factor of $m: m$. Therefore, gcd($a,m$) = 1 for all $0<a<m$.
    
    We will next prove $\phi\left(m\right) = m-1 \rightarrow$ $m$ is prime. Since every number $a$ s.t. $0<a<m$ can be written as a product of 1 or more primes $a_1,a_2,\dots ,a_n < a$, and $m$ can be written as a product of 1 or more primes $m_1,m_2,\dots ,m_s < m$. Since $m$ shares no factors with $0<a<m$ and $0<m_i\leq m$, that leave m as the only choice for $m_i$. Therefore the only factor of $m$ is $m$, so $m$ is prime. 
    \\
    
\noindent\textbf{1.25}
    Use the Square and Multiply algorithm to compute the following:
    \begin{enumerate}[label=(\alph*)]
        \item $17^{183} \equiv \fbox{129} \Mod{256}$
        \item $2^{477} \equiv \fbox{984} \Mod{1000}$
        \item $11^{507} \equiv \fbox{1013} \Mod{1237}$
    \end{enumerate}
    Code: (I decided to code the mod in by hand instead of utilizing the ring of integers mod m for better understanding of what the program was doing)
    \singlespacing
    \begin{lstlisting}
        p=2
        A=1000
        m=100
        binary = bin(A)
        a = [i for i in binary]
        a = a[2:]
        answer = 1
        for i in a:
            if i == '0':
                answer = (answer^2)%m
            if i == '1':
                answer = ((answer*p)^2)%m
        print(answer)
    \end{lstlisting} 
    \doublespacing
\noindent\textbf{1.26} (a)
    Let $P = \{ p_1,p_2,\dots ,p_r\}$ be a set of prime numbers, and let $N = p_1p_2\dots p_r + 1$. Prove that N is divisible by some prime not in P. Then deduce that there are an infinite number of primes
    
    We will prove this using tools of modular arithmetic. For N to be divisible by a prime $p\in P$, $N \equiv 0 \Mod{p}$. 
    $$N\equiv p_1p_2\dots p_r + 1 \Mod{p}$$
    which is equivalent to $1 \Mod{p}$ since $p_1p_2\dots p_r$ is divisible by $p$ since $p\in P$. Since $1 \not\equiv 0 \Mod{p}$. Therefore, since $p$ is an arbitrary element of $P$, N is not divisible by any $p\in P$.\\
    
    We can let $P =$ \{finite set of all primes\}. Now, $N= $ "product of all primes" $+ 1$. Which we proved is not divisible by any prime in P. Which contradicts the fundamental theorem of arithmetic. Therefore P is not finite, or P is infinite. Therefore, there are an infinite number of primes.\\
    

\noindent\textbf{1.27}
    Without using the fact that every integer has a unique factorization into primes, prove that if gcd($a,b$)$=1$ and if $a\mid bc$, then $a\mid c$.\\
    
    Using the fact that $a \mid bc$ implies that $a \mid b$ or $a \mid c$. Since gcd(a,b) = 1 by assumption, $a \not\mid b$. Therefore, $a \mid c$

\noindent\textbf{1.33}
    Let $p$ be a prime such that $q = \frac{1}{2}\left(p-1\right)$ is also prime. Suppose that $g$ is an integer satisfying $g \not\equiv\pm 1 \Mod{p}$ and $g^q \not\equiv 1 \Mod{p}$.
    Prove that g is a primitive root modulo p\\
    
    This is identical to proving that $\mid g\mid = \mid U_p\mid$ Due to lagrange's theorem, $\mid g\mid$ is a positive factor of $\mid U_p\mid$. Since $\mid U_p\mid = p-1$ ($p$ is prime), $q=\frac{1}{2}(p-1)=\mid U_p\mid /2$. Since $p-1$ is always even, this number is whole. If $g^q \not\equiv 1 \Mod{p}$, then the order of $g$ is not a factor of $q$, therefore the only factor of $\mid U_p\mid$ that is not a factor of $q$ is $\mid U_p\mid$ itself. Therefore the order of $g$ is equal to the order of $U_p$. So g $g$ is a primitive root modulo p.

\end{document}
