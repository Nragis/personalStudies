\documentclass[letterpaper, 12pt]{article}
\usepackage{comment} % enables the use of multi-line comments (\ifx \fi) 
\usepackage{fullpage} % changes the margin
\usepackage[T1]{fontenc}
\usepackage{selinput}
\usepackage{enumitem}
\usepackage{listings}
\usepackage{amsmath}
\usepackage{amssymb}
\usepackage{setspace}
\usepackage{bm}
\usepackage{mathtools}

\newcommand{\Mod}[1]{\ (\mathrm{mod}\ #1)}

\begin{document}


%Header-Make sure you update this information!!!!
\noindent
\large\textbf{SP: Probability and Computing} \hfill \textbf{Quinn Murphey} \\
\normalsize MAT 4953.001 \hfill Date: 02/24/21\\
Dr. Ergur\hfill Due Date: 02/24/21\\
\noindent\makebox[\linewidth]{\rule{\paperwidth}{0.4pt}}


\section*{HW 2\#:}
	\noindent\textbf{2.7) Let X and Y be independent geometric random variables with parameters p and q respectively. Find the values of the following}
	\begin{enumerate}
		\item Pr$(X = Y)$.

			\begin{align*}
				\text{Pr}(X=Y) &= \sum_{n=1}^\infty \text{Pr}(X=n \land Y=n) \\
				&= \sum_{n=1}^\infty \text{Pr}(X=n)\text{Pr}(Y=n) \\
				&= \sum_{n=1}^\infty (1-p)^{n-1}p\cdot(1-q)^{n-1}q \\
				&= pq\sum_{n=1}^\infty ((1-p)(1-q))^{n-1} \\
				&= \frac{pq}{1 - (1-p)(1-q)} = \frac{pq}{p + q - pq}.
			\end{align*}

		\item $\mathbb{E}[\text{max}(X,Y)]$.

			\begin{align*}
				\mathbb{E}[\text{max}(X,Y)] &= \sum_{n=1}^\infty \text{Pr}(\text{max}(X,Y)\geq n) \text{Lem 2.9} \\
				&= \sum_{n=1}^\infty \text{Pr}(X\geq n) + \text{Pr}(Y\geq n) - \text{Pr}(X\geq n \land Y\geq n) \\
				&= \sum{n=1}^\infty (1-p)^{n-1} + (1-q)^{n-1} - (1-p)^{n-1}(1-q)^{n-1} \\
				&= \frac{1}{1 - (1-p)} + \frac{1}{1 - (1-q)} - \frac{1}{1 - (1-p)(1-q)} \\
				&= \frac{1}{p} + \frac{1}{q} - \frac{1}{p + q - pq}.
			\end{align*}
			
		\item Pr$(\text{min}(X,Y) = k)$.

			\begin{align*}
				\text{Pr}(\text{min}(X,Y) = k) &= \text{Pr}(X = k \land Y>k) + \text{Pr}(Y = k \land x \geq k) \\
				&= \text{Pr}(X=k)\text{Pr}(Y>k) + \text{Pr}(y=k)\text{Pr}(x\geq Y) \\
				&= (1-p)^{k-1}p(1-q)^{k-1} + (1-q)^{k-1}q(1-p)^{k-1}\\
				&= (1-p)^{k-1}(1-q)^{k-1}(p(1-q) + q) \\
				&= (1 - (p + q - pq))^{k-1}(p + q -pq)
			\end{align*}

		\item $\mathbb{E}[X \mid X\leq Y]$.

			\begin{align*}
				\mathbb{E}[X\mid X\leq Y] &= \sum_{n=1}^{\infty}n\cdot\text{Pr}(X=n\mid X\leq Y) \\
                &= \sum_{n=1}^\infty n \cdot \frac{\text{Pr}(X=n \land X\leq Y}{\text{Pr}(X\leq Y)} \\
			\end{align*}
			
			We will first evaluate the inner parts of this sum.

			\begin{align*}
				\text{Pr}(X\leq Y) &= \sum_{n=1}^{\infty} \text{Pr}(X=n \land n\leq Y) \\
				&= \sum_{n=1}^\infty \text{Pr}(X=n)\text{Pr}(n\leq Y) \\
				&= \sum_{n=1}^\infty (1-p)^{n-1}p(1-q)^{n-1} \\
				&= \frac{p}{1 - (1-p)(1-q)} = \frac{p}{p + q - pq}.
			\end{align*}

			\begin{align*}
				\text{Pr}(X=n \land n\leq Y) &= \text{Pr}(X=n)\text{Pr}(n\leq Y) \\
				&= (1-p)^{n-1}p(1-q)^{n-1}.
			\end{align*}

			Thus,

			\begin{align*}
				\mathbb{E}[X\mid X\leq Y] &= \sum_{n=1}^{\infty}n\cdot\frac{(1-p)^{n-1}p(1-q)^{n-1}}{\frac{p}{p + q - pq}} \\
				&= \sum_{n=0}^\infty n\cdot (1-(p + q - pq))^{n-1}(p+q-pq) \\
				&= \frac{1}{p + q - pq}
			\end{align*}

			since the second line is the expectation of a geometric random variable with parameter $p+q-pq$.
	\end{enumerate}

	\noindent\textbf{2.9) Suppose we roll two fair k-sided dice to get the values $X_1$ and $X_2$. Find the following values:}
	\begin{enumerate}
		\item $\mathbb{E}[\text{max}(X_1,X_2)]$

			\begin{align*}
				\mathbb{E}[\text{max}(X_1,X_2)] &= \sum_{i=1}^{k}\text{Pr}(\text{max}(X_1,X_2)\geq i) \\
				&= \sum_{i=1}^k \text{Pr}(X_1 \geq i) + \text{Pr}(X_2 \geq i) = \text{Pr}(X_1 \geq i \land X_2\geq i) \\
				&= \sum_{i=1}^k (\frac{i}{k}) + (\frac{i}{k}) - (\frac{i}{k})(\frac{i}{k}) \\
				&= \sum_{i=1}^k 2\frac{i}{k} - \frac{i^2}{k^2} \\
				&= \frac{k(k+1)}{k} - \frac{k(k+1)(2k+1)}{6k^2} \\
				&= k+1 - \frac{(k+1)(2k+1)}{6k}.
			\end{align*}

		\item $\mathbb{E}[\text{max}(X_1,X_2)$
	

			Note: $\mathbb{E}[X_1] = \mathbb{E}[X_2] = \frac{k+1}{2}$ by multiplying the sum of the first $k$ numbers by the probability of each roll: $\frac{1}{k}$.

			\begin{align*}
				\mathbb{E}[\text{min}(X_1,X_2)] &= \mathbb{E}[X_1 + X_2 - \text{max}(X_1,X_2)] \\
				&= \mathbb{E}[X_1] + \mathbb{E}[X_2] - \mathbb{E}[\text{max}(X_1,X_2)] \text{ (linearity of E)}\\
				&= \frac{k+1}{2} \frac{k+1}{2} - (k+1 - \frac{(k+1)(2k+1)}{6k}).
			\end{align*}

	\end{enumerate}

	(b) and (c) follow directly from how we defined min$(X_1, X_2)$ in the beginning of its evaluation and the fact that max$(X,Y) = X + Y -$ min$(X,Y)$.

	\noindent\textbf{2.15) Let $X = X_1 + \dots + X_k$ where each $X_i$ is a geometric random variable with parameter $p$. Find $\mathbb{E}[X]$ }

	By the linearity of expectation we get $\mathbb{E}[X] = \sum_{i=1}^k\mathbb{E}[X_i] = k\cdot p$.

	\noindent\textbf{2.25) When rolling one six sided dice at a time, what is the expected number of rolls until consecutive sixes are shown}
	
	Let $X$ denote the random variable on the number of rolls before consecutive sixes are rolled and $X_1$ a geometric random variable with $p=1/6$.

	We first note that if we roll a six, then we have $1/6$ probability of rolling consecutive sixes on the next roll and a 5/6 probability of not. This means our next roll can be treated as the first roll of a new run because it's independent from the rolls that came before. However, if we don't roll 6, then, there is no possibility of rolling consecutive sixes on the next roll.

	Formalizing this, we get

	\begin{align*}
		\mathbb{E}[X\mid X_1] &= \frac{1}{6}(X_1 + 1) + \frac{5}{6}(X_1 + 1 + \mathbb{E}[X]) \\
		&= X_1 + 1 +\frac{5}{6}\mathbb{E}[X].
	\end{align*}

	We can then solve for $\mathbb{E}$ to get
	\begin{align*}
		\mathbb{E}[X] &= \mathbb{E} \mathbb{E} [X \mid X_1] \\
		&= \mathbb{E}[X_1 + 1 \mathbb{E}[X]] \\
		&= \mathbb{E}[X] = \mathbb{E}[X_1] + 1 + \frac{5}{6}\mathbb{E}[X] \\
		&\Rightarrow \frac{1}{6}\mathbb{E}[X] = 7 \\
		&\Rightarrow \mathbb{E}[X] = 42.
	\end{align*}
	
	Thus, we expect it will take 42 rolls to roll consecutive sixes
	
\end{document}
